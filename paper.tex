% Template LaTeX menggunakan forlimits.cls
\documentclass[final]{forlimits}

%%-----berikut ini diisi oleh redaksi---------------------------------
\journalinfo{xx}{x}{Bulan Tahun} {h--h} {http://dx.doi.org/10.12962/limits.v19i1.11060}
\kodenaskah{LM-1234XX}
\date{\tanggalpengajuan{dd-mm-yyyy}{dd-mm-yyyy}{dd-mm-yyyy}}
%%-----------------------------------------------------------------

%---- dari sini penulisan naskah Anda dimulai --------------------------
\title{Petunjuk Penggunaan Kelas LaTeX untuk Menulis Artikel pada Jurnal Limits dengan Berbagai Pilihan Format}

%nama-nama penulis, tambahkan tanda * di belakang nama penulis korespondensi (tanpa spasi) 
\author[1]{Amat Cendekia}
\author[2]{Bahagia Sentosa Raja*}
\author[3]{Cekatan Gajah Perkasa}
\author[2]{Dermawan~Kayaraya}

%--afiliasi penulis
\affil[1]{Departemen Matematika, Institut Teknologi Sepuluh Nopember, Surabaya}
\affil[2]{Prodi. Sains Komputasi, Jurusan Matematika, Universitas Gajah Perkasa, Jogjakarta}
\affil[3]{Prodi. Komputasi Statistika, Departemen Matematika, Universitas Permata Cendekia, Bali}

%---email hanya diisi untuk penulis korespondensi (corresponding author)
\correspondingauthor{Bahagia S.R.}{bahagia@contoh.its.ac.id}

% Mengisi short authors dan short title secara manual
\shortauthors{A Cendekia, BS Raja, CG Perkasa, D Kayaraya} % Nama pendek untuk header
\shorttitle{Petunjuk Menulis Artikel untuk Jurnal Limits} % Judul pendek untuk header

\begin{document}
	
	\maketitle
	
	\begin{abstrak}
		Abstrak ditulis di sini dalam bahasa Indonesia. Sebisa mungkin, abstrak tidak memuat simbol atau ekspresi matematika. Abstrak juga tidak  mencantumkan perujukan (sitasi) maupun gambar. Pastikan bahwa abstrak dalam bahasa Indonesia dan Inggris tidak melebihi halaman pertama.
	\end{abstrak}
	\katakunci{kata pertama, kata kedua, kata ketiga, kata keempat}
	
	\begin{abstract}
		Abstract is written here in English. As much as possible, the abstract should not contain symbols or mathematical expressions. It should also not include citations nor figures. Ensure that the abstract in both Indonesian and English does not exceed the first page.
	\end{abstract}
	\keywords{first word, second word, third word, last word}
	

	
	\section{Pendahuluan}
	Isi pendahuluan dapat terdiri dari latar belakang dan motivasi penelitian. Selain itu, dapat juga diisi dengan teori pendukung yang digunakan dalam pembahasan. 
	
	\subsection{Beberapa Contoh}
	\paragraph{Menulis teorema dan buktinya.}~ 
	Untuk menulis teorema ({\it theorem}) atau sejenisnya, yaitu lemma ({\it lemma}), proposisi ({\it proposition}), dan akibat ({\it corollary})) dituliskan dengan perintah, misalnya seperti berikut:
	\begin{theorem}\label{Teo:01}
		Ini adalah teorema yang diberi label {\normalfont{\texttt{teo:01}}}  yang digunakan untuk merujuk dengan perintah  \verb|Teorema~\ref{T:01}| atau \verb|\autoref{T:01}| dari bagian lain dalam dokumen  ini. Sebagai contoh, lihat pada penulisan bukti berikut ini.  
	\end{theorem}
	
	\begin{proof}
	Berdasarkan hipotesis pada \autoref{T:01} dapat dipastikan tampilannya cukup sesuai dengan yang diharapkan. Jika tidak, bisa digunakan Teorema~\ref{T:01} sebagai alternatif. 
	\end{proof}
	
	\section{Definisi dan Notasi}
	Isi metode penelitian...
	\begin{example}
		 Perhatikan persamaan berikut ini:
		 \[ F(t) := \int_a^t \frac{x^2-\sin(x)}{\sin(x)-e^{2x}}\,dx =
		 \begin{pmatrix}
		 	\frac{dS}{dt} & \frac{dR}{dt} & \frac{dT}{dt}\\
		 	\sin(a)            & \cos(a)          & 1\\
		 	S(a)               & R(a)               & T(a) 
		 \end{pmatrix}
		 \]
\end{example}


	\section{Hasil dan Pembahasan}
	Isi hasil dan pembahasan...
	\begin{definition}\cite{jurnal11}
	Hasil ini dipenuhi, jika 
	\begin{equation}\label{Pers:01}
		F(t) := \begin{pmatrix}
			S(t) & R(t) &T(t) \\ a & 1 & 1\\a^2 & a & 1
		\end{pmatrix}
	\end{equation}
	untuk $\mathcal{S}(t)$ atau $\mathscr{S}(t)$ berdistribusi acak. Sedangkan $\mathcal{R}(t)$ atau $\mathscr{R}(t)$ bergantung pada $t$ awal di $\mathbb{R}$ positif.
	\end{definition}
	
Untuk menuliskan ekspresi matematika yang terdiri dari beberapa baris dengan satu nomor label, ditulis seperti berikut (salah satu caranya):
\begin{align}
	\int_a^t f(x,y)\, dx & = \int_a^{t_0} \frac{\partial f(x,y)}{\partial x}\,dy \nonumber\\
	                               & = \int_b^{t_0} \frac{\partial f(x,y)}{\partial y}\,dx \label{Pers:mult}
\end{align}

Untuk memastikan validitas \autoref{Pers:mult} banyak yang bisa dikerjakan untuk itu, khususnya menggunakan \eqref{Pers:01}.  Contoh penulisan ekspresi matematika dalam baris teks $L=\int_a^b f(x)\,dx$ dan penulisan matriks kecil $ \left( \begin{smallmatrix} a & b \\ c & d \end{smallmatrix} \right) $ dalam baris yang bersamaan dengan teks kalimat. Contoh lain, untuk menulis persamaan matematika yang memuat bentuk pecahan $\frac{dy}{dx}$ atau $\frac{\partial f}{\partial y}$ atau bentuk limit $\lim_{x\to a} f(x)=L$. Dituliskannya seperti itu, tidak ditulis dengan $\dfrac{dy}{dx}$ atau $\dfrac{\partial f}{\partial y}$; juga tidak dengan penulisan $\lim\limits_{x\to a}f(x)=L$, karena akan menghasilkan tampilan yang kurang rapi, yaitu jarak antar baris menjadi tidak konsisten. Jika ingin menampilkan tulisan yang lebih besar, lebih baik ditulis $dy/dx$ atau $\partial f/\partial y$.  

Berikut ini contoh penulisan tabel dengan {\it caption} yang benar.

\begin{table}[!ht]
	\centering
	\caption{Contoh penulisan tabel}
	\label{Tbl:01}
	\begin{tabularx}{\textwidth}{rlXX}\hline
		1. & Namanya SIapaya & Perjalanan sengaja ada juga yang mengira jauh& Tiada sebanding majalah tugas menjadi\\
		2. & Namanya Dianya & Mengapa ada yang tidak Perjalanan sengaja ada juga & Tiada sebanding majalah tugas kaya raya imannya\\
		\vdots & \vdots \\\hline 
	\end{tabularx} 
\end{table} 
Banyak yang bisa dituliskan di sini. Banyak yang bisa dituliskan di sini. Banyak yang bisa dituliskan di sini. Banyak yang bisa dituliskan di sini. Banyak yang bisa dituliskan di sini. Banyak yang bisa dituliskan di sini. Banyak yang bisa dituliskan di sini. Banyak yang bisa dituliskan di sini. Banyak yang bisa dituliskan di sini. Banyak yang bisa dituliskan di sini. Banyak yang bisa dituliskan di sini. Banyak yang bisa dituliskan di sini. 

\begin{theorem}\label{T:01}
	Ini teorema pertama dalam artikel ini, dan ini merupakan hasil utama.
\end{theorem}
\begin{proof}
	Ini bagian bukti dari \autoref{T:01} dan merupakan uaraian lengkap dan terperinci dari hasil utama dari penelitian ini. 
\end{proof}

Selanjutnya digunakan definisi berikut ini untuk menguraikan beberapa temuan selanjutnya. 
\begin{definition}\label{Def:01}
Definisi baru yang digunakan didalam penelitian ini.
\end{definition}

Banyak yang bisa dituliskan di sini. Banyak yang bisa dituliskan di sini. Banyak yang bisa dituliskan di sini. Banyak yang bisa dituliskan di sini. Banyak yang bisa dituliskan di sini. Banyak yang bisa dituliskan di sini. Banyak yang bisa dituliskan di sini. Banyak yang bisa dituliskan di sini. Banyak yang bisa dituliskan di sini. Banyak yang bisa dituliskan di sini. Banyak yang bisa dituliskan di sini. Banyak yang bisa dituliskan di sini. Banyak yang bisa dituliskan di sini. 

\[ F(x,y) := \iint_\mathbb{R} \frac{\sin(x^2+y^2)\cdot e^{\sin{x^2}}\,dx\,dy}{\cos(e^{x+y})}  =\oint f(x,y)\,dt. \]
Banyak yang bisa dituliskan di sini.  Banyak yang bisa dituliskan di sini. Banyak yang bisa dituliskan di sini. Banyak yang bisa dituliskan di sini. Banyak yang bisa dituliskan di sini. Banyak yang bisa dituliskan di sini. Banyak yang bisa dituliskan di sini. Banyak yang bisa dituliskan di sini. Banyak yang bisa dituliskan di sini. Banyak yang bisa dituliskan di sini. Banyak yang bisa dituliskan di sini. Banyak yang bisa dituliskan di sini.  Banyak yang bisa dituliskan di sini.  Banyak yang bisa dituliskan di sini. Banyak yang bisa dituliskan di sini. Banyak yang bisa dituliskan di sini. 
	
	\section{Kesimpulan}
	Tuliskan kesimpulan di bagian ini, yaitu berisi ringkasan hasil utama, implikasi dari hasil tersebut, serta arah penelitian selanjutnya. 


\terimakasih{Penulis pertama dalam artikel ini mendapat bantuan dana dari Hibah Penelitian Kolaborasi Internasional nomor 789/UN1/DITLIT/2035.}

%---Daftar Pustaka
\begin{thebibliography}{99}
	
	\bibitem{buku1}
	Nama Penulis.
	\textit{Judul Buku}.
	Edisi ke-2. Kota: Penerbit, 2023.
	
	\bibitem{jurnal1}
	Nama Penulis.
	``Judul Artikel Jurnal,'' 
	\textit{Nama Jurnal}, vol. 10, no. 2, hlm. 123--134, 2021.
	
	\bibitem{jurnal2}
	Nama Penulis Satu dan Nama Penulis Dua.
	``Judul Artikel Jurnal Lain,'' 
	\textit{Nama Jurnal}, vol. 10, no. 2, hlm. 123--456, 2025.
	
	\bibitem{internet}
	Nama Penulis atau Organisasi.
	``Judul Halaman Web.'' 
	Diakses dari: \texttt{https://www.example.com/artikel}, 
	diakses pada 10 April 2024.
	
	\bibitem{tesis1}
	Nama Penulis.
	\textit{Judul Tesis}.
	Tesis (S2), Nama Universitas, Kota, 2020.
	
	\bibitem{disertasi1}
	Nama Penulis.
	\textit{Judul Disertasi}.
	Disertasi (S3), Nama Universitas, Kota, 2022.
	
	\bibitem{konferensi1}
	Nama Penulis.
	``Judul Makalah Konferensi,'' 
	dalam \textit{Prosiding Konferensi Nasional Nama Konferensi}, 
	Kota, 2019, hlm. 45--50.
	
	\bibitem{laporan1}
	Nama Penulis atau Lembaga.
	\textit{Judul Laporan}.
	Laporan Teknis No. 12-2021, Kota: Instansi, 2021.
	
	\bibitem{standar1}
	Nama Organisasi Standar.
	\textit{Judul Standar Teknis}.
	Kode Standar, Tahun.
	
	\bibitem{skripsi1}
	Nama Penulis.
	\textit{Judul Skripsi}.
	Skripsi (S1), Nama Universitas, Kota, 2018.
	
	\bibitem{jurnal11}
	Nama Penulis1, Nama Penulis2, dan Nama Penulis3.
	``Judul Artikel Jurnal,'' 
	\textit{Nama Jurnal}, vol. 19, no. 4, hlm. 345--354, 2021.
\end{thebibliography}
	
\end{document}
