% Template LaTeX menggunakan forlimits.cls
\documentclass[final]{forlimits}
\usepackage{graphicx}
\usepackage{amsmath}
\usepackage{float}
\usepackage{hyperref}
\usepackage{enumitem}
\usepackage{titlesec}
\usepackage{setspace}
\usepackage[utf8]{inputenc}
\usepackage{tabularx} % Untuk tabel yang lebarnya bisa disesuaikan
\usepackage{multirow} % Untuk menggabungkan baris
\usepackage{booktabs} % Untuk garis tabel yang lebih bagus


%%-----berikut ini diisi oleh redaksi---------------------------------
\journalinfo{xx}{x}{Bulan Tahun} {h--h} {http://dx.doi.org/10.12962/limits.v19i1.11060}
\kodenaskah{LM-1234XX}

% \date{\tanggalpengajuan{15-01-2005}{dd-mm-yyyy}{dd-mm-yyyy}}
%%-----------------------------------------------------------------

%---- dari sini penulisan naskah Anda dimulai --------------------------
\title{JUDUL}

%nama-nama penulis, tambahkan tanda * di belakang nama penulis korespondensi (tanpa spasi) 
\author[1]{Raihan Putra Akbar}
\author[2]{Gerald Eberhard}
\author[3]{Muhammad Irfan Wira Kusuma}

%--afiliasi penulis
\affil[1,2,3]{Program Studi Ilmu Komputer, Fakultas Sains dan Ilmu Komputer, Universitas Pertamina, Jakarta}

%---email hanya diisi untuk penulis korespondensi (corresponding author)
\correspondingauthor{Raihan P.A.}{105223001@student.universitaspertamina.ac.id}

% Mengisi short authors dan short title secara manual
\shortauthors{RP Akbar, G Eberhard, MIW Kusuma} % Nama pendek untuk header
\shorttitle{Petunjuk Menulis Artikel untuk Jurnal Limits} % Judul pendek untuk header

\begin{document}
	
	\maketitle
	
	\begin{abstrak}
		Abstrak ditulis di sini dalam bahasa Indonesia. Sebisa mungkin, abstrak tidak memuat simbol atau ekspresi matematika. Abstrak juga tidak  mencantumkan perujukan (sitasi) maupun gambar. Pastikan bahwa abstrak dalam bahasa Indonesia dan Inggris tidak melebihi halaman pertama.
	\end{abstrak}
	\katakunci{kata pertama, kata kedua, kata ketiga, kata keempat}
	
	\begin{abstract}
		Abstract is written here in English. As much as possible, the abstract should not contain symbols or mathematical expressions. It should also not include citations nor figures. Ensure that the abstract in both Indonesian and English does not exceed the first page.
	\end{abstract}
	\keywords{first word, second word, third word, last word}
	

	
	\section{Pendahuluan}
	Pergerakan harga saham merupakan cerminan harapan pasar terhadap kondisi ekonomi dan penilaian terhadap prospek kinerja perusahaan(REF). Dalam konteks perekonomian nasional, indeks dan saham sektor perbankan sering dipandang sebagai indikator kesehatan sistem keuangan karena peran bank sebagai perantara utama dalam penyaluran modal dan likuiditas (REF). Perbankan memegang posisi strategis dalam struktur ekonomi sehingga fluktuasi yang signifikan pada saham-saham perbankan dapat berdampak pada  risiko keuangan, likuiditas pasar, dan kebijakan pemerintah. Untuk itu, analisis terhadap saham-saham perbankan besar memberi wawasan yang penting bagi pemangku kebijakan yang membutuhkan sinyal-sinyal awal gangguan di sektor keuangan.

	Dalam studi ini, fokus pembahasan ditujukan pada tiga saham perbankan terbesar di Indonesia, yaitu BBCA (Bank BCA), BBRI (Bank BRI), dan BMRI (Bank Mandiri). Saham-saham ini dipilih karena sebagai top 3 terbesar \textit{market capitalization} dalam situs IDX pada tahun 2025 (REF). Hal ini diasumsikan sebagai indikator kapitalisasi pasar dan likuiditas yang menjadi representatif untuk analisis perilaku perbankan. Pemilihan saham-saham ini bertujuan untuk melakukan kajian komparatif terhadap penurunan harga yang disebabkan oleh faktor makro atau bersifat internal yang terjadi pada masing-masing perusahaan. Penelitian ini mencoba melihat pola umum yang mungkin muncul dari dinamika permintaan-penawaran, volatilitas pasar, dan arus modal asing yang masuk atau keluar dari saham-saham perbankan tersebut.

	Analisis penyebab penurunan harga saham merupakan tantangan tersendiri karena penurunan dapat berasal dari faktor internal, seperti perubahan manajemen atau masalah tata kelola, maupun faktor eksternal, seperti kondisi makroekonomi atau perubahan regulasi. Pendekatan yang digunakan adalah penggunaan data historis dan teknik analisis \textit{time-series} sehingga memungkinkan pemisahan komponen perubahan harga  sehingga dapat memberikan indikator apakah perubahan bersifat sektoral atau bersifat individual. Oleh karena itu, fokus penelitian diarahkan pada analisis historis dan deskriptif.

	Analisis \textit{time-series} memungkinkan analisis tren jangka panjang melalui teknik \textit{smoothing} seperti \textit{moving averages}, deteksi musim (\textit{seasonality}), dan pengukuran volatilitas melalui perhitungan return harian. Selain itu, indikator teknis seperti \textit{Relative Strength Index} (RSI) dipakai sebagai fitur tambahan untuk menentukan kondisi \textit{overbought/oversold} yang membantu interpretasi pola pergerakan harga.

	Sebagai langkah awal, \textit{Exploratory Data Analysis} (EDA) digunakan untuk memahami distribusi variabel, mendeteksi outlier, membuat ringkasan statistik, dan mengungkap pola yang tersirat. Dalam penelitian ini, alur yang digunakan meliputi \textit{preprocessing} data dengan konversi waktu dan penanganan \textit{missing value}. Lalu pembuatan fitur, seperti SMA, RSI, \textit{daily return}, dan \textit{moving average volume} untuk masing-masing saham. Selanjutnya, dilakukan visualisasi untuk membandingkan harga dan arus modal, serta dekomposisi \textit{time-series} untuk memisahkan komponen \textit{trend}, \textit{seasonality}, dan \textit{residual}. Langkah-langkah tersebut dilakukan untuk dapat diinterpretasikan menjadi dasar analisis penurunan harga yang bersifat sektoral atau bersifat individual.

	Penelitian ini bersifat eksploratif dan deskriptif dengan tujuan utama menyajikan analisis komparatif mengenai saham BBCA, BBRI, dan BMRI sehingga memberikan \textit{insight} awal bagi analisis risiko sektoral. Kontribusi praktis penelitian meliputi rangkaian prosedur EDA yang dapat direplikasi untuk saham lain untuk memahami hubungan antara harga, volatilitas, volume, dan aliran modal asing.

	\section{Dataset dan Metodologi}

	\subsection{Deskripsi Dataset}

	Dataset yang digunakan dalam penelitian ini merupakan data historis perdagangan saham di Bursa Efek Indonesia (BEI) dengan fokus pada tiga emiten utama, yaitu Bank Central Asia Tbk. (BBCA), Bank Rakyat Indonesia Tbk. (BBRI), dan Bank Mandiri Tbk. (BMRI). Dataset diperoleh dengan mengakses \textit{website} IDX dan mengambil data json yang kemudian diolah menjadi format XLSX untuk memudahkan analisis dan proses data (REF MANUAL BOOK).
	
	Data yang diambil bersifat harian dan mencakup informasi harga, volume perdagangan, serta aktivitas transaksi investor asing. Pemilihan data harian bertujuan untuk menangkap dinamika pasar secara lebih granular, termasuk fluktuasi jangka pendek, volatilitas, serta respon pasar terhadap perubahan sentimen. Periode waktu yang dicakup dalam dataset ini adalah dari tanggal 8 November 2021 hingga 6 Januari 2026, sehingga mencakup lebih dari empat tahun data historis yang memungkinkan analisis tren jangka panjang serta pola musiman.

	\begin{figure}[H]
        \centering
        \includegraphics[width=1.0\textwidth]{img/ContohPreviewData.png}         
		\caption{Preview Dataset Saham Perbankan}
        \label{fig:Preview Dataset Saham Perbankan}
    \end{figure}

	Berdasarkan Gambar \ref{fig:Preview Dataset Saham Perbankan}, dataset memuat berbagai atribut yang berkaitan dengan aktivitas perdagangan saham, mulai dari identitas emiten, informasi harga, volume transaksi, hingga data penawaran (\textit{bid-offer}) dan transaksi non-reguler. Namun, tidak seluruh kolom digunakan dalam analisis ini. Penelitian ini menitikberatkan pada kolom-kolom yang secara langsung merepresentasikan pergerakan harga, likuiditas pasar, dan arus modal asing.  Variabel-variabel tersebut relevan untuk analisis time-series dan eksplorasi pola pergerakan saham.

	Deskripsi singkat mengenai kolom-kolom utama yang digunakan dalam penelitian ini disajikan pada Tabel \ref{tab:deskripsi_kolom} berikut
	\begin{table}[H]
		\centering
		\caption{Deskripsi Kolom Dataset Saham}
		\label{tab:deskripsi_kolom}
		\renewcommand{\arraystretch}{1.2}
		\begin{tabularx}{\textwidth}{|>{\hsize=.3\hsize}X|>{\hsize=.2\hsize}X|>{\hsize=.5\hsize}X|} 
			\hline
			\textbf{Kategori} & \textbf{Nama Kolom} & \textbf{Deskripsi} \\
			\hline
			\multirow{3}{=}{Informasi waktu dan identitas saham} 
			& \texttt{Date} & Tanggal perdagangan, digunakan sebagai indeks time-series \\ \cline{2-3}
			& \texttt{StockCode} & Kode saham (BBCA, BBRI, BMRI) \\ \cline{2-3}
			& \texttt{StockName} & Nama emiten \\ \hline
			
			\multirow{6}{=}{Informasi harga saham}
			& \texttt{OpenPrice} & Harga pembukaan \\ \cline{2-3}
			& \texttt{High} & Harga tertinggi harian \\ \cline{2-3}
			& \texttt{Low} & Harga terendah harian \\ \cline{2-3}
			& \texttt{Close} & Harga penutupan \\ \cline{2-3}
			& \texttt{Previous} & Harga penutupan hari sebelumnya \\ \cline{2-3}
			& \texttt{Change} & Perubahan harga penutupan harian \\ \hline
			
			\multirow{3}{=}{Informasi aktivitas perdagangan} 
			& \texttt{Volume} & Jumlah saham yang diperdagangkan \\ \cline{2-3}
			& \texttt{Value} & Nilai transaksi perdagangan \\ \cline{2-3}
			& \texttt{Frequency} & Frekuensi transaksi \\ \hline
			
			\multirow{2}{=}{Informasi arus modal asing} 
			& \texttt{ForeignBuy} & Volume pembelian oleh investor asing \\ \cline{2-3}
			& \texttt{ForeignSell} & Volume penjualan oleh investor asing \\ \hline
		\end{tabularx}
	\end{table}
	
	\subsection{Beberapa Contoh}
	\paragraph{Menulis teorema dan buktinya.}~ 
	Untuk menulis teorema ({\it theorem}) atau sejenisnya, yaitu lemma ({\it lemma}), proposisi ({\it proposition}), dan akibat ({\it corollary})) dituliskan dengan perintah, misalnya seperti berikut:
	\begin{theorem}\label{Teo:01}
		Ini adalah teorema yang diberi label {\normalfont{\texttt{teo:01}}}  yang digunakan untuk merujuk dengan perintah  \verb|Teorema~\ref{T:01}| atau \verb|\autoref{T:01}| dari bagian lain dalam dokumen  ini. Sebagai contoh, lihat pada penulisan bukti berikut ini.  
	\end{theorem}
	
	\begin{proof}
	Berdasarkan hipotesis pada \autoref{T:01} dapat dipastikan tampilannya cukup sesuai dengan yang diharapkan. Jika tidak, bisa digunakan Teorema~\ref{T:01} sebagai alternatif. 
	\end{proof}
	
	\section{Definisi dan Notasi}
	Isi metode penelitian...
	\begin{example}
		 Perhatikan persamaan berikut ini:
		 \[ F(t) := \int_a^t \frac{x^2-\sin(x)}{\sin(x)-e^{2x}}\,dx =
		 \begin{pmatrix}
		 	\frac{dS}{dt} & \frac{dR}{dt} & \frac{dT}{dt}\\
		 	\sin(a)            & \cos(a)          & 1\\
		 	S(a)               & R(a)               & T(a) 
		 \end{pmatrix}
		 \]
\end{example}


	\section{Hasil dan Pembahasan}
	Isi hasil dan pembahasan...
	\begin{definition}\cite{jurnal11}
	Hasil ini dipenuhi, jika 
	\begin{equation}\label{Pers:01}
		F(t) := \begin{pmatrix}
			S(t) & R(t) &T(t) \\ a & 1 & 1\\a^2 & a & 1
		\end{pmatrix}
	\end{equation}
	untuk $\mathcal{S}(t)$ atau $\mathscr{S}(t)$ berdistribusi acak. Sedangkan $\mathcal{R}(t)$ atau $\mathscr{R}(t)$ bergantung pada $t$ awal di $\mathbb{R}$ positif.
	\end{definition}
	
Untuk menuliskan ekspresi matematika yang terdiri dari beberapa baris dengan satu nomor label, ditulis seperti berikut (salah satu caranya):
\begin{align}
	\int_a^t f(x,y)\, dx & = \int_a^{t_0} \frac{\partial f(x,y)}{\partial x}\,dy \nonumber\\
	                               & = \int_b^{t_0} \frac{\partial f(x,y)}{\partial y}\,dx \label{Pers:mult}
\end{align}

Untuk memastikan validitas \autoref{Pers:mult} banyak yang bisa dikerjakan untuk itu, khususnya menggunakan \eqref{Pers:01}.  Contoh penulisan ekspresi matematika dalam baris teks $L=\int_a^b f(x)\,dx$ dan penulisan matriks kecil $ \left( \begin{smallmatrix} a & b \\ c & d \end{smallmatrix} \right) $ dalam baris yang bersamaan dengan teks kalimat. Contoh lain, untuk menulis persamaan matematika yang memuat bentuk pecahan $\frac{dy}{dx}$ atau $\frac{\partial f}{\partial y}$ atau bentuk limit $\lim_{x\to a} f(x)=L$. Dituliskannya seperti itu, tidak ditulis dengan $\dfrac{dy}{dx}$ atau $\dfrac{\partial f}{\partial y}$; juga tidak dengan penulisan $\lim\limits_{x\to a}f(x)=L$, karena akan menghasilkan tampilan yang kurang rapi, yaitu jarak antar baris menjadi tidak konsisten. Jika ingin menampilkan tulisan yang lebih besar, lebih baik ditulis $dy/dx$ atau $\partial f/\partial y$.  

Berikut ini contoh penulisan tabel dengan {\it caption} yang benar.

\begin{table}[!ht]
	\centering
	\caption{Contoh penulisan tabel}
	\label{Tbl:01}
	\begin{tabularx}{\textwidth}{rlXX}\hline
		1. & Namanya SIapaya & Perjalanan sengaja ada juga yang mengira jauh& Tiada sebanding majalah tugas menjadi\\
		2. & Namanya Dianya & Mengapa ada yang tidak Perjalanan sengaja ada juga & Tiada sebanding majalah tugas kaya raya imannya\\
		\vdots & \vdots \\\hline 
	\end{tabularx} 
\end{table} 
Banyak yang bisa dituliskan di sini. Banyak yang bisa dituliskan di sini. Banyak yang bisa dituliskan di sini. Banyak yang bisa dituliskan di sini. Banyak yang bisa dituliskan di sini. Banyak yang bisa dituliskan di sini. Banyak yang bisa dituliskan di sini. Banyak yang bisa dituliskan di sini. Banyak yang bisa dituliskan di sini. Banyak yang bisa dituliskan di sini. Banyak yang bisa dituliskan di sini. Banyak yang bisa dituliskan di sini. 

\begin{theorem}\label{T:01}
	Ini teorema pertama dalam artikel ini, dan ini merupakan hasil utama.
\end{theorem}
\begin{proof}
	Ini bagian bukti dari \autoref{T:01} dan merupakan uaraian lengkap dan terperinci dari hasil utama dari penelitian ini. 
\end{proof}

Selanjutnya digunakan definisi berikut ini untuk menguraikan beberapa temuan selanjutnya. 
\begin{definition}\label{Def:01}
Definisi baru yang digunakan didalam penelitian ini.
\end{definition}

Banyak yang bisa dituliskan di sini. Banyak yang bisa dituliskan di sini. Banyak yang bisa dituliskan di sini. Banyak yang bisa dituliskan di sini. Banyak yang bisa dituliskan di sini. Banyak yang bisa dituliskan di sini. Banyak yang bisa dituliskan di sini. Banyak yang bisa dituliskan di sini. Banyak yang bisa dituliskan di sini. Banyak yang bisa dituliskan di sini. Banyak yang bisa dituliskan di sini. Banyak yang bisa dituliskan di sini. Banyak yang bisa dituliskan di sini. 

\[ F(x,y) := \iint_\mathbb{R} \frac{\sin(x^2+y^2)\cdot e^{\sin{x^2}}\,dx\,dy}{\cos(e^{x+y})}  =\oint f(x,y)\,dt. \]
Banyak yang bisa dituliskan di sini.  Banyak yang bisa dituliskan di sini. Banyak yang bisa dituliskan di sini. Banyak yang bisa dituliskan di sini. Banyak yang bisa dituliskan di sini. Banyak yang bisa dituliskan di sini. Banyak yang bisa dituliskan di sini. Banyak yang bisa dituliskan di sini. Banyak yang bisa dituliskan di sini. Banyak yang bisa dituliskan di sini. Banyak yang bisa dituliskan di sini. Banyak yang bisa dituliskan di sini.  Banyak yang bisa dituliskan di sini.  Banyak yang bisa dituliskan di sini. Banyak yang bisa dituliskan di sini. Banyak yang bisa dituliskan di sini. 
	
	\section{Kesimpulan}
	Tuliskan kesimpulan di bagian ini, yaitu berisi ringkasan hasil utama, implikasi dari hasil tersebut, serta arah penelitian selanjutnya. 


\terimakasih{Penulis pertama dalam artikel ini mendapat bantuan dana dari Hibah Penelitian Kolaborasi Internasional nomor 789/UN1/DITLIT/2035.}

%---Daftar Pustaka
\begin{thebibliography}{99}
	
	\bibitem{buku1}
	Nama Penulis.
	\textit{Judul Buku}.
	Edisi ke-2. Kota: Penerbit, 2023.
	
	\bibitem{jurnal1}
	Nama Penulis.
	``Judul Artikel Jurnal,'' 
	\textit{Nama Jurnal}, vol. 10, no. 2, hlm. 123--134, 2021.
	
	\bibitem{jurnal2}
	Nama Penulis Satu dan Nama Penulis Dua.
	``Judul Artikel Jurnal Lain,'' 
	\textit{Nama Jurnal}, vol. 10, no. 2, hlm. 123--456, 2025.
	
	\bibitem{internet}
	Nama Penulis atau Organisasi.
	``Judul Halaman Web.'' 
	Diakses dari: \texttt{https://www.example.com/artikel}, 
	diakses pada 10 April 2024.
	
	\bibitem{tesis1}
	Nama Penulis.
	\textit{Judul Tesis}.
	Tesis (S2), Nama Universitas, Kota, 2020.
	
	\bibitem{disertasi1}
	Nama Penulis.
	\textit{Judul Disertasi}.
	Disertasi (S3), Nama Universitas, Kota, 2022.
	
	\bibitem{konferensi1}
	Nama Penulis.
	``Judul Makalah Konferensi,'' 
	dalam \textit{Prosiding Konferensi Nasional Nama Konferensi}, 
	Kota, 2019, hlm. 45--50.
	
	\bibitem{laporan1}
	Nama Penulis atau Lembaga.
	\textit{Judul Laporan}.
	Laporan Teknis No. 12-2021, Kota: Instansi, 2021.
	
	\bibitem{standar1}
	Nama Organisasi Standar.
	\textit{Judul Standar Teknis}.
	Kode Standar, Tahun.
	
	\bibitem{skripsi1}
	Nama Penulis.
	\textit{Judul Skripsi}.
	Skripsi (S1), Nama Universitas, Kota, 2018.
	
	\bibitem{jurnal11}
	Nama Penulis1, Nama Penulis2, dan Nama Penulis3.
	``Judul Artikel Jurnal,'' 
	\textit{Nama Jurnal}, vol. 19, no. 4, hlm. 345--354, 2021.
\end{thebibliography}
	
\end{document}
